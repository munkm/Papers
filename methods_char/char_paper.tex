\documentclass{article}                                                                           %basic LaTeX document type
%%%%%%%%%%%%%%%%%%%%%%%%%%%%%%%%%%%%%%%%%%%%%%%%%%%%%%%%%%%%%%%%%%%%%

\usepackage[margin=1.3in]{geometry}                                                     %set size of all margins
%\usepackage[left=1.3in, top=1in, right=1in, bottom=1in]{geometry}         %can set margin sizes which are not the same in this way

%\linespread{2}                                                                                         %option 1 for making text double-spaced
\usepackage{setspace}                                                                             %option 2 for making text double-spaced
\doublespacing                                                                                         %set double space

\usepackage{indentfirst}                                                                          %makes first paragraph of section indented (non-first are by default)
\setlength{\parindent}{25pt}                                                                     %set size of indentation (15pt is default)

\renewcommand\thesection{\Roman{section}}                                         %set capital Roman numeral section headings
\renewcommand\thesubsection{\thesection.\Alph{subsection}}                   %set capital Aramaic letters subsection headings
\renewcommand\thesubsubsection{\thesubsection.\arabic{subsubsection}} %set capital Arabic numbers subsubsection headings

\renewcommand*\thetable{\Roman{table}}                                              %set capital Roman numeral table numeration

\usepackage[explicit]{titlesec}                                                                  %package needed for next lines
\titleformat{\section}{\bfseries}{\thesection.}{1em}{\MakeUppercase{#1}}  %makes section headings bold and upper case characters
\titleformat{\subsection}{\bfseries}{\thesubsection.}{1em}{#1}                   %makes subsection headings bold
\titleformat{\subsubsection}{\itshape}{\thesubsubsection.}{1em}{#1}          %makes subsubsection headings in italics

\usepackage{lastpage}                                                                             %package which returns number of last page (same as number of pages)
\usepackage[figure,table]{totalcount}                                                        %package which counts the number of tables and/or figures

\usepackage{amsmath}                                                                            %enable `align' equation types

\renewcommand{\thefootnote}{\alph{footnote}}                                        %sets labeling of footnotes
\usepackage[]{footmisc}                                                                          %enables double spaced footnotes
  \renewcommand{\footnotelayout}{\doublespacing}                                  %double spacing of footnotes

%%% If you desire to place footnotes at the end of the document, uncomment these
%%%  four lines and the two related near the end of this file
%\usepackage{endnotes}                                                                          %enables endnote page
%  \let\footnote=\endnote                                                                         %sets footnotes to endnotes
%  \renewcommand*{\theendnote}{\alph{endnote}}                                   %labels endnotes alphabetically
%  \renewcommand{\notesname}{Footnotes}                                            %renames endnotes title to footnotes

\usepackage{graphicx}                                                                             %enable figures
\usepackage{multirow}                                                                             %enable `multirow' capability in tables

\usepackage{caption}                                                                              %enables subfigures
\usepackage[labelformat=simple]{subcaption}                                           %enables subfigure captions
\captionsetup[table]{labelsep=newline,name=TABLE}                                 %sets table caption formatting options to meet NSE requirements
\captionsetup[figure]{name=Fig.,labelsep=period}                                     %sets figure caption options to meet NSE requirements
\renewcommand*\thesubfigure{(\alph{subfigure})}                                    %enables proper labeling of subfigures

\usepackage[colorlinks=true, allcolors=blue]{hyperref}                               %reference and citation hyperlinking linking
%%%%%%%%%%%%%%%%%%%%%%%%%%%%%%%%%%%%%%%%%%%%%%%%%%%%%%%%%%%%%%%%%%%%%

\DeclareMathOperator{\diff}{d}                                                                %can define operators to use in math mode, example in Eq. 2
\DeclareMathOperator{\erf}{erf}
%%%%%%%%%%%%%%%%%%%%%%%%%%%%%%%%%%%%%%%%%%%%%%%%%%%%%%%%%%%%%%%%%%%%%

\begin{document}

%Define fields for \maketitle

\title{CADIS-$\Omega$: Theory and Characterization of an Angle-Informed Hybrid
Method for Deep-Penetration Radiation Transport} %title of paper

\author{
\vspace{20mm}
Madicken Munk,$^{\text{a},\ast}$ Rachel N. Slaybaugh$^{\text{a}}$ \\[4pt] %list of authors, with corresponding author marked by asterisk
\textit{$^a$Department of Nuclear Engineering, University of California,
Berkeley}\\[-10pt]       %affiliations of authors
\textit{4133 Etcheverry Hall, MC 1730, University of California, Berkeley,}\\ [-10pt]
\textit{Berkeley, CA 94720-1730} \\ [-5pt]
\textit{$^b$Radiation Transport and Nuclear Systems Division, Oak Ridge National
Laboratory} \\ [-10pt]
\textit{Oak Ridge National Laboratory} \\ [-10pt]
{$^\ast$Email: \href{mailto:madicken@berkeley.edu}{madicken@berkeley.edu}}}       % email address for correspondence

\date{                               %instead of returning the date, this repurposes the \maketitle command to print the number of pages, tables, and figures
\vspace{40mm}
Number of pages: \pageref*{LastPage} \\
Number of tables: \totaltables \\
Number of figures: \totalfigures
}

\clearpage\maketitle
\thispagestyle{empty}


\pagebreak
~\vfill

%%%%%%%%%%%%%%%%%%%%%%%%%%%%%%%%%%%%%%%%%%%%%%%%%%%%%%%%%%%%%%%%%%%%%

\begin{abstract}
This is the abstract

\vspace{1em}\noindent\textbf{Keywords} ---
hybrid methods, radiation shielding, radiation transport
\end{abstract}

\vfill


%%%%%%%%%%%%%%%%%%%%%%%%%%%%%%%%%%%%%%%%%%%%%%%%%%%%%%%%%%%%%%%%%%%%%

\pagebreak
\section{Introduction}

The usefulness of a template is increased if examples are given of many things that an author may encounter.
For example, the introduction to the paper goes here in the first section.
Let's use this introduction as a chance to cite different types of references including a book~\cite{Pomraning1991book}, a non-ANS-produced journal paper~\cite{AdamsJQSRT1989}, an ANS transactions paper~\cite{ZimmermanANS1991}, an ANS-produced journal paper~\cite{DonovanNSE2003}, a master's thesis~\cite{Vasques2005thesis}, a dissertation~\cite{Fichtl2009dissertation}, a conference paper~\cite{BrantleyMC2009Incident}, and a national laboratory report~\cite{PautzSAND2017AMClosurePres}.
These publications have been selected to demonstrate how to cite different types of works---not to provide an exhaustive introduction to the stochastic media problem in radiation transport.\footnote{Some of these references have more mature counterparts, e.g., Ref.~\cite{BrantleyMC2009Incident} is a conference paper whose content is encompassed by and expanded on in a journal paper.}

A seminal book on the problem of radiation transport in stochastic media was published in the early 1990s~\cite{Pomraning1991book}.
This body of work yielded a highly referenced journal paper presenting the ``Levermore--Pomraning'' (LP) closure~\cite{AdamsJQSRT1989}.
Soon thereafter a Monte Carlo algorithm was proposed by Zimmerman and Adams (Ref.~\cite{ZimmermanANS1991}) which has been shown to produce results equivalent to those given by the LP closure, as well as an algorithm that is usually more accurate.
Various extensions to these Monte Carlo algorithms have been investigated including variants intended for use with non-Markovian media~\cite{DonovanNSE2003}.
Work on this topic has continued, including in the academic realm, yielding a master's thesis which was a review of the dominant methods at that time~\cite{Vasques2005thesis} and a dissertation investigating an entirely new approach~\cite{Fichtl2009dissertation}.
One outcome of a newer body of work was to produce Monte-Carlo-based benchmarks~\cite{BrantleyMC2009Incident}.
Researchers continue to develop new approaches to solving transport results in stochastic media problems, as evidenced by Ref.~\cite{PautzSAND2017AMClosurePres} (a corresponding conference paper exists).

The next section details the use of equations, while Sec.~\ref{sec:tables} provides examples of table usage.
Secs.~\ref{sec:figures} and~\ref{sec:sections} contain example usages of figures and subheadings, respectively.
We conclude with a link to the ANS submission guidelines in Sec.~\ref{sec:conclusion}.



%%%%%%%%%%%%%%%%%%%%%%%%%%%%%%%%%%%%%%%%%%%%%%%%%%%%%%%%%%%%%%%%%%%%%

\section{Equations}
\label{sec:equations}

Equations are a part of the grammatical syntax of a sentence and may be a vital part of an independent clause; Eq.~(\ref{eqn:chordlength}) provides an example of this.
Ref.~\cite{AdamsJQSRT1989} explains that in binary Markovian media, the probability of finding material \(i\) at any point in the rod is given by
\begin{equation}
  p_i = \frac{\lambda_i}{\lambda_0 + \lambda_1}.
  \label{eqn:chordlength}
\end{equation}
Equations may also be a parenthetical statement to further explain the meaning of a sentence but are not required to form an independent clause.
This sentence and the following equation, which states the Levermore--Pomraning equations as written in Ref.~\cite{AdamsJQSRT1989}, provides an example:
\begin{align}
\begin{split}
  v \frac{\partial(p_i \psi_i)}{\partial t} + \Omega \nabla(p_i \psi_i) + \sigma_i p_i \psi_i &= \left[ \frac{\sigma_{si}}{4\pi} \right] \int_{4\pi} \diff \Omega' p_i \psi_i(\Omega') + p_i S_i + \frac{p_j \psi_j}{\lambda_j} - \frac{p_i \psi_i}{\lambda_i}, \\
  & i, j = 0,1, \qquad j \neq i.
\end{split}
  \label{eqn:LPequations}
\end{align}



%%%%%%%%%%%%%%%%%%%%%%%%%%%%%%%%%%%%%%%%%%%%%%%%%%%%%%%%%%%%%%%%%%%%%

\section{Tables}
\label{sec:tables}

Tables ought to be formatted to span either one or two columns and referenced
specifically in the text.
For example, Tables~\ref{tab:params} and~\ref{tab:results}, based off of
tables in Ref.~\cite{BrantleyMC2009Incident}, describe the input parameters and some numerical
results for benchmark cases first presented in Ref.~\cite{AdamsJQSRT1989}.
Table~\ref{tab:params} is an example of a one-column table, and Table~\ref{tab:results}
is an example of a two-column table.

\begin{table}%[!htbp]
\centering
\caption{Benchmark Set Parameters}
\label{tab:params}
\begin{tabular}{|ccccc|}
\hline
    Case & \(\Sigma_{t,0}\) & \(\Sigma_{t,1}\) & \(p_0\) & \(\lambda_c\) \\ \hline
    1               & 10/99 & 100/11 & 0.9 & 0.099\\
    2               & 10/99 & 99/10   & 0.9 & 0.99 \\
    3               & 2/101 & 200/101  & 0.5 & 2.525 \\ \hline
\hline
    Case  & $c_0$ & $c_1$ &  \multicolumn{2}{|c|}{Slab Thickness}  \\ \hline
    a               & 0.0 & 1.0 & \multicolumn{1}{|c}{\(L=\)} & 0.1 \\
    b               & 1.0 & 0.0 & \multicolumn{1}{|c}{\(L=\)} & 1.0 \\
    c               & 0.9 & 0.9 & \multicolumn{1}{|c}{\(L=\)} & 10.0 \\ \hline
\end{tabular}
\end{table}

\begin{table}[b]%[!htbp]
\centering
\caption{Reflection and Transmission Probability Comparisons}
\label{tab:results}
\begin{tabular}{|c|c|c|c|c|c|c|c|}
\hline
            &         &              &                 &                   &                  & \multicolumn{2}{c|}{Relative Error $E_{\langle R \rangle,\langle T \rangle}$} \\\hline
    \(L\)   & Case & Quantity & Benchmark & Algorithm A & Algorithm B & Algorithm A & Algorithm B \\ \hline \hline
    \multirow{4}{*}{0.1}   & \multirow{2}{*}{a} & \(\langle R \rangle\) & 0.04864 & 0.04768 & 0.04876 & -0.020 & 0.002 \\
                                     &                              & \(\langle T \rangle\) & 0.93650 & 0.93463 & 0.93350 & -0.002 & -0.003 \\ \cline{2-8}
                                     & \multirow{2}{*}{b} & \(\langle R \rangle\) & 0.00868 & 0.00847 & 0.00860 & -0.024 & -0.009 \\
                                     &                              & \(\langle T \rangle\) & 0.90432 & 0.90062 & 0.90080 & -0.004 & -0.004 \\ \hline
\end{tabular}
\end{table}

%%%%%%%%%%%%%%%%%%%%%%%%%%%%%%%%%%%%%%%%%%%%%%%%%%%%%%%%%%%%%%%%%%%%%

\section{Figures}
\label{sec:figures}


%%%%%%%%%%%%%%%%%%%%%%%%%%%%%%%%%%%%%%%%%%%%%%%%%%%%%%%%%%%%%%%%%%%%%

\section{Sections}
\label{sec:sections}

You will likely want to use subsections to keep the paper organized and easy to read.

\subsection{Example 1 of B-level Heading}

Subsections are formatted in this way.

\subsection{Example 2 of B-level Heading}

If you have one subsection, ideally you should have at least two.

\subsubsection{Example 1 of C-level Heading}

Subsubsections may also be used.

\subsubsection{Example 2 of C-level Heading}

But again, it is preferred to have at least two.



%%%%%%%%%%%%%%%%%%%%%%%%%%%%%%%%%%%%%%%%%%%%%%%%%%%%%%%%%%%%%%%%%%%%%

\section{Conclusion}
\label{sec:conclusion}

We hope this template serves you well.
For the current information on ANS document formatting, visit
\url{http://www.ans.org/pubs/journals/nse/authors/} for the NSE journal
or the equivalent sites for the
\href{http://www.ans.org/pubs/journals/nt/authors/}{NT}
and \href{http://www.ans.org/pubs/journals/fst/authors/}{FST} journals.

\pagebreak
\section*{Acknowledgments}

Template creation supported by the Department of Energy Computational Science Graduate Fellowship, provided under grant number DE-FG02-97ER25308.

\pagebreak
\bibliographystyle{ans_js}                                                                           %custom ANS journal submission template bibliography style
\bibliography{bibliography}

%%% If you desire to place footnotes at the end of the document, uncomment these two lines and the four related near the beginning of this file
%\pagebreak
%\theendnotes

\end{document}


