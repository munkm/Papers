\documentclass{article}                                                                           %basic LaTeX document type
%%%%%%%%%%%%%%%%%%%%%%%%%%%%%%%%%%%%%%%%%%%%%%%%%%%%%%%%%%%%%%%%%%%%%

\usepackage[margin=1.3in]{geometry}                                                     %set size of all margins
%\usepackage[left=1.3in, top=1in, right=1in, bottom=1in]{geometry}         %can set margin sizes which are not the same in this way

%\linespread{2}                                                                                         %option 1 for making text double-spaced
\usepackage{setspace}                                                                             %option 2 for making text double-spaced
\doublespacing                                                                                         %set double space

\usepackage{indentfirst}                                                                          %makes first paragraph of section indented (non-first are by default)
\setlength{\parindent}{25pt}                                                                     %set size of indentation (15pt is default)

\renewcommand\thesection{\Roman{section}}                                         %set capital Roman numeral section headings
\renewcommand\thesubsection{\thesection.\Alph{subsection}}                   %set capital Aramaic letters subsection headings
\renewcommand\thesubsubsection{\thesubsection.\arabic{subsubsection}} %set capital Arabic numbers subsubsection headings

\renewcommand*\thetable{\Roman{table}}                                              %set capital Roman numeral table numeration

\usepackage[explicit]{titlesec}                                                                  %package needed for next lines
\titleformat{\section}{\bfseries}{\thesection.}{1em}{\MakeUppercase{#1}}  %makes section headings bold and upper case characters
\titleformat{\subsection}{\bfseries}{\thesubsection.}{1em}{#1}                   %makes subsection headings bold
\titleformat{\subsubsection}{\itshape}{\thesubsubsection.}{1em}{#1}          %makes subsubsection headings in italics

\usepackage{lastpage}                                                                             %package which returns number of last page (same as number of pages)
\usepackage[figure,table]{totalcount}                                                        %package which counts the number of tables and/or figures

\usepackage{amsmath}                                                                            %enable `align' equation types

\renewcommand{\thefootnote}{\alph{footnote}}                                        %sets labeling of footnotes
\usepackage[]{footmisc}                                                                          %enables double spaced footnotes
  \renewcommand{\footnotelayout}{\doublespacing}                                  %double spacing of footnotes

%%% If you desire to place footnotes at the end of the document, uncomment these
%%%  four lines and the two related near the end of this file
%\usepackage{endnotes}                                                                          %enables endnote page
%  \let\footnote=\endnote                                                                         %sets footnotes to endnotes
%  \renewcommand*{\theendnote}{\alph{endnote}}                                   %labels endnotes alphabetically
%  \renewcommand{\notesname}{Footnotes}                                            %renames endnotes title to footnotes

\usepackage{graphicx}                                                                             %enable figures
\usepackage{multirow}                                                                             %enable `multirow' capability in tables

\usepackage{caption}                                                                              %enables subfigures
\usepackage[labelformat=simple]{subcaption}                                           %enables subfigure captions
\captionsetup[table]{labelsep=newline,name=TABLE}                                 %sets table caption formatting options to meet NSE requirements
\captionsetup[figure]{name=Fig.,labelsep=period}                                     %sets figure caption options to meet NSE requirements
\renewcommand*\thesubfigure{(\alph{subfigure})}                                    %enables proper labeling of subfigures

\usepackage[colorlinks=true, allcolors=blue]{hyperref}                               %reference and citation hyperlinking linking
%%%%%%%%%%%%%%%%%%%%%%%%%%%%%%%%%%%%%%%%%%%%%%%%%%%%%%%%%%%%%%%%%%%%%

\DeclareMathOperator{\diff}{d}                                                                %can define operators to use in math mode, example in Eq. 2
\DeclareMathOperator{\erf}{erf}
%%%%%%%%%%%%%%%%%%%%%%%%%%%%%%%%%%%%%%%%%%%%%%%%%%%%%%%%%%%%%%%%%%%%%

\begin{document}

%Define fields for \maketitle

\title{CADIS-$\Omega$: Theory and Characterization of an Angle-Informed Hybrid
Method for Deep-Penetration Radiation Transport} %title of paper

\author{
\vspace{20mm}
Madicken Munk,$^{\text{a},\ast}$ Rachel N. Slaybaugh$^{\text{a}}$ \\[4pt] %list of authors, with corresponding author marked by asterisk
\textit{$^a$Department of Nuclear Engineering, University of California,
Berkeley}\\[-10pt]       %affiliations of authors
\textit{4133 Etcheverry Hall, MC 1730, University of California, Berkeley,}\\ [-10pt]
\textit{Berkeley, CA 94720-1730} \\ [-5pt]
\textit{$^b$Radiation Transport and Nuclear Systems Division, Oak Ridge National
Laboratory} \\ [-10pt]
\textit{Oak Ridge National Laboratory} \\ [-10pt]
{$^\ast$Email: \href{mailto:madicken@berkeley.edu}{madicken@berkeley.edu}}}       % email address for correspondence

\date{                               %instead of returning the date, this repurposes the \maketitle command to print the number of pages, tables, and figures
\vspace{40mm}
Number of pages: \pageref*{LastPage} \\
Number of tables: \totaltables \\
Number of figures: \totalfigures
}

\clearpage\maketitle
\thispagestyle{empty}


\pagebreak
~\vfill

%%%%%%%%%%%%%%%%%%%%%%%%%%%%%%%%%%%%%%%%%%%%%%%%%%%%%%%%%%%%%%%%%%%%%

\begin{abstract}
This is the abstract

\vspace{1em}\noindent\textbf{Keywords} ---
hybrid methods, radiation shielding, radiation transport
\end{abstract}

\vfill


%%%%%%%%%%%%%%%%%%%%%%%%%%%%%%%%%%%%%%%%%%%%%%%%%%%%%%%%%%%%%%%%%%%%%

\section{Introduction}
\label{sec:introduction}


\section{Theoretical Background}
\label{sec:background}


\section{$\Omega$-Method Theory}
\label{sec:theory}


\section{Characterization of the Method}
\label{sec:characterization}


\section{Summary of Observed Phenomena}
\label{sec:observations}


\section{Conclusion}
\label{sec:conclusions}



%%%%%%%%%%%%%%%%%%%%%%%%%%%%%%%%%%%%%%%%%%%%%%%%%%%%%%%%%%%%%%%%%%%%%


\pagebreak
\section*{Acknowledgments}

Template creation supported by the Department of Energy Computational Science Graduate Fellowship, provided under grant number DE-FG02-97ER25308.

\pagebreak
\bibliographystyle{ans_js}                                                                           %custom ANS journal submission template bibliography style
\bibliography{bibliography}

%%% If you desire to place footnotes at the end of the document, uncomment these two lines and the four related near the beginning of this file
%\pagebreak
%\theendnotes

\end{document}


