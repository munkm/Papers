
\documentclass[12pt]{article}

\usepackage{physor2016}

\usepackage{bm}

\usepackage{graphicx}
\usepackage{tikz}
\usepgflibrary{shapes.geometric}

\usepackage{booktabs}

\usepackage{siunitx}
%------------------------------------------------------------------------------

%------------------------------------------------------------------------------
% Define title. Use all CAPITALS.
%------------------------------------------------------------------------------
\title{AN ANGLE-INFORMED HYBRID METHOD FOR DEEP-PENETRATION RADIATION TRANSPORT APPLIED TO CADIS AND FW-CADIS}
%
% ...and authors
%
\author{ 
  \textbf{Madicken Munk and R.\ N.\ Slaybaugh} \\
  Department of Nuclear Engineering, University of California, Berkeley \\
  3115B Etcheverry Hall, Berkeley, CA 94720\\
  \href{mailto:madicken@berkeley.edu}{madicken@berkeley.edu}\\
  \href{mailto:slaybaugh@berkeley.edu}{slaybaugh@berkeley.edu}\\
  \\
  \textbf{T.\ Pandya, S.\ Johnson, T.\ M.\ Evans}\\
  Radiation Transport Group\\
  Oak Ridge National Laboratory, P.O.\ Box 2008, Oak Ridge, TN 37831\\
  \href{mailto:evanstm@ornl.gov}{evanstm@ornl.gov}
  }
  
%------------------------------------------------------------------------------
\renewcommand{\shortauthor}      % Author's names here
           {M.\ Munk~et~al.}  
\renewcommand{\shorttitle}       % Short title here
           {Angle-Informed CADIS and FW-CADIS}  

%------------------------------------------------------------------------------
% Setup PDF info. This sets several values which are listed as the "properties"
% of the PDF file.
%------------------------------------------------------------------------------
\hypersetup{
  pdftitle=\shorttitle,
  pdfauthor=\shortauthor
}


\begin{document}

%\doublespacing

%\linenumbers

%------------------------------------------------------------------------------
% Make the titlepage and set the pagestyle to fancy throughout
%------------------------------------------------------------------------------
\maketitle

\begin{abstract}
This is our abstract
\end{abstract}

\keywords{Hybrid Methods, CADIS, FW-CADIS, Angular Biasing}

%------------------------------------------------------------------------------
%
%------------------------------------------------------------------------------
\section{INTRODUCTION}
\label{sect::intro}

Motivate hybrid methods (problems we want to solve and can't or are hard)\\
We have established a new method to take on this problem; brief description of method \\
Briefly how this method is novel compared to other methods\\
Preview of results (tell 'em the punchline)\\
Outline of paper (state what will be in each section)


%------------------------------------------------------------------------------
%
%------------------------------------------------------------------------------
\section{BACKGROUND}
\label{sect::second}

Intro paragraph describing what is covered in background and why\\
(if makes sense for flow/content) quick statement about types of VR and what we'll focus on)\\
Write forward and adjoint importance equation and define terms\\
explain Adjoint = Importance; point out this is a concept frequently used in variance reduction \\
expalin Contributon = Response; give any context about use of contributons \\

\section{PAST WORK}
\label{sec::past}
Highlight the CADIS method \\
Highlight the FW-CADIS method \\
Talk about what problems CADIS and FW-CADIS succeed in, then talk about where they fall short. \\
Introduce angle-informed and angle-biased methods. \\
Higlight AVATAR \\
Highlight Turner \& Larsen's method. \\
Conclude section with discussion of what is missing from CADIS/FW-CADIS/Angle Methods. \\

%------------------------------------------------------------------------------
%
%------------------------------------------------------------------------------
\section{METHODOLOGY}
\label{sect::second}

Introduce section with goal of our method. \\

%------------------------------------------------------------------------------
%
%------------------------------------------------------------------------------
\subsection{Theory}
\label{subsect::major}

Show our method equation. \\
Discuss how this relates to the concept of importance with the adjoint. \\
Discuss how this relates to contributon response. \\
Say physically how this helps us track in problems with strong anisotropy. \\
Highlight how this is different from past methods.

%------------------------------------------------------------------------------
%
%------------------------------------------------------------------------------

\subsection{Implementation}
\label{subsect::major}

Introduce code packages we used. \\
Write brief intro for MCNP, ADVANTG, Denovo. \\


%------------------------------------------------------------------------------
%
%------------------------------------------------------------------------------
\subsubsection{Implementation in Denovo} 
\label{subsubsect::minor}

Describe what was done in Denovo. \\
Motivate how any user could use what is available in Denovo to then use our method. \\


%------------------------------------------------------------------------------
%
%------------------------------------------------------------------------------
\subsubsection{Implementation in ADVANTG} 
\label{subsubsect::minor}

Describe packages in ADVANTG that were created. 


%------------------------------------------------------------------------------
%
%------------------------------------------------------------------------------
\section{RESULTS AND DISCUSSION} 
\label{sect::references}

Show test problem geometry
\\ Show flux map from analog MC
\\ Show flux uncertainties from analog MC
\\ Show biasing parameter map from CADIS
\\ Show flux map from CADIS
\\ Show flux uncertainties from CADIS
\\ Show Contributon Distributon from new method
\\ Show biasing parameter distribution from new method
\\ Show flux map from new method
\\ Show flux uncertainties from new method
\\ Discuss differences and meaning of differences of results and parameter maps
\\ Discuss implications of these differences for solving other problems more generally

%------------------------------------------------------------------------------
%
%------------------------------------------------------------------------------
\section{FUTURE WORK} 
\label{sect::references}

Show the future test phase space we plan to cover. 


%------------------------------------------------------------------------------
%
%------------------------------------------------------------------------------
\section{CONCLUSION} 
\label{sect::conclusion}

Recap what we told them (what problem we're trying to solve; about this method; how it's different than past methods)\\
Recap problem we solved and results\\
Strong punchline paragraph about potential impact based on theory and these results.

%------------------------------------------------------------------------------
%
%------------------------------------------------------------------------------
\section*{ACKNOWLEDGMENTS}

This material is based on work supported by the Department of Energy under award number DE-NE0008286. This report was prepared as an account of work sponsored by an agency of the United States Government. Neither the United States Government nor any agency thereof, nor any of their employees, makes any warranty, express or implied, or assumes any legal liability or responsibility for the accuracy, completeness, or usefulness of any information, apparatus, product, or process disclosed, or represents that its use would not infringe privately owned rights. Reference herein to any specific commercial product, process, or service by trade name, trademark, manufacturer, or otherwise does not necessarily constitute or imply its endorsement, recommendation, or favoring by the United States Government or any agency thereof. The views and opinions of the authors expressed herein do not necessarily state or reflect those of the United States Government or any agency thereof.

\bibliographystyle{physor2016}
\bibliography{physor2016}

\appendix

\makeatletter
\def\@seccntformat#1{APPENDIX \csname the#1\endcsname.~}
\makeatother

%------------------------------------------------------------------------------
% If you need to make one (or more) appendix (appendices), place them here as
% sections
%------------------------------------------------------------------------------
\section{HOW TO MAKE APPENDICES}
\label{app::a}

This is a placeholder for my first appendix

\section{OTHER APPENDIX STUFF}
\label{app::b}

This is a placeholder for my second appendix

\end{document}

