
\documentclass[12pt]{article}

\usepackage{physor2016}

\usepackage{bm}

\usepackage{graphicx}
\usepackage{tikz}
\usepgflibrary{shapes.geometric}

\usepackage{booktabs}

\usepackage{siunitx}
%------------------------------------------------------------------------------

%------------------------------------------------------------------------------
% Define title. Use all CAPITALS.
%------------------------------------------------------------------------------
\title{AN ANGLE-INFORMED HYBRID METHOD FOR DEEP-PENETRATION RADIATION TRANSPORT APPLIED TO CADIS AND FW-CADIS}
%
% ...and authors
%
\author{ 
  \textbf{Madicken Munk and Rachel N. Slaybaugh} \\
  Department of Nuclear Engineering \\
  University of California, Berkeley \\
  3115B Etcheverry Hall, Berkeley, CA \\
  \href{mailto:madicken@berkeley.edu}{madicken@berkeley.edu}\\
  }


%------------------------------------------------------------------------------
\renewcommand{\shortauthor}      % Author's names here
           {M. Munk and R. N. Slaybaugh}  
\renewcommand{\shorttitle}       % Short title here
           {Angle-Informed CADIS and FW-CADIS}  

%------------------------------------------------------------------------------
% Setup PDF info. This sets several values which are listed as the "properties"
% of the PDF file.
%------------------------------------------------------------------------------
\hypersetup{
  pdftitle=\shorttitle,
  pdfauthor=\shortauthor
}


\begin{document}

%\doublespacing

%\linenumbers

%------------------------------------------------------------------------------
% Make the titlepage and set the pagestyle to fancy throughout
%------------------------------------------------------------------------------
\maketitle

\begin{abstract}
This is our abstract
\end{abstract}

\keywords{Hybrid Methods, CADIS, FW-CADIS, Angular Biasing}

%------------------------------------------------------------------------------
%
%------------------------------------------------------------------------------
\section{INTRODUCTION}
\label{sect::intro}

Discuss importance of hybrid methods \\
Establish existing state of commonly used hybrid methods \\
Mention that angle-informed methods exist \\
Briefly mention why our method helps to contribute to this. \\


%------------------------------------------------------------------------------
%
%------------------------------------------------------------------------------
\section{BACKGROUND}
\label{sect::second}

Talk about why hybrid methods are important (for deep-penetration RT) \\
Highlight how Adjoint = Importance \\
Highlight how Contributon = Response \\
Highlight the CADIS method \\
Highlight the FW-CADIS method \\
Talk about what probelms CADIS and FW-CADIS succeed in, then talk about where they fall short. \\
Introduce angle-informed and angle-biased methods. \\
Higlight AVATAR \\
Highlight Turner \& Larsen's method. \\
Conclude section with discussion of what is missing from CADIS/FW-CADIS/Angle Methods. \\

%------------------------------------------------------------------------------
%
%------------------------------------------------------------------------------
\section{METHODOLOGY}
\label{sect::second}

Introduce section with goal of our method. \\

%------------------------------------------------------------------------------
%
%------------------------------------------------------------------------------
\subsection{Theory}
\label{subsect::major}

Show our method equation. \\
Discuss how this relates to the concept of importance with the adjoint. \\
Discuss how this relates to contributon response. \\
Say physically how this helps us track in problems with strong anisotropy. \\

%------------------------------------------------------------------------------
%
%------------------------------------------------------------------------------

\subsection{Implementation}
\label{subsect::major}

Introduce code packages we used. \\
Write brief intro for MCNP, ADVANTG, Denovo. \\


%------------------------------------------------------------------------------
%
%------------------------------------------------------------------------------
\subsubsection{Implementation in Denovo} 
\label{subsubsect::minor}

Describe what was done in Denovo. \\
Motivate how any user could use what is available in Denovo to then use our method. \\


%------------------------------------------------------------------------------
%
%------------------------------------------------------------------------------
\subsubsection{Implementation in ADVANTG} 
\label{subsubsect::minor}

Describe packages in ADVANTG that were created. 


%------------------------------------------------------------------------------
%
%------------------------------------------------------------------------------
\section{RESULTS AND DISCUSSION} 
\label{sect::references}

Show test problem geometry
\\ Show flux map for analog MC
\\ Show flux uncertainties for analog
\\ Show flux map for FW-CADIS MC
\\ Show flux uncertanties for FW-CADIS
\\ Show Contributon Distributon for geometry
\\ Show biasing parameter distribution for geometry
\\ Show flux map for our method
\\ Show flux uncertanties for our method
\\ Discuss differences and issues encountered with each

%------------------------------------------------------------------------------
%
%------------------------------------------------------------------------------
\section{FUTURE WORK} 
\label{sect::references}

Show the future test phase space we plan to cover. 


%------------------------------------------------------------------------------
%
%------------------------------------------------------------------------------
\section{CONCLUSION} 
\label{sect::conclusion}

Write concluding remarks on what we've done so far and why we anticipate this being significant. 

%------------------------------------------------------------------------------
%
%------------------------------------------------------------------------------
\section*{ACKNOWLEDGMENTS}

This is where we will thank our money- and knowledge- suppliers. 


\bibliographystyle{physor2016}
\bibliography{physor2016}

\appendix

\makeatletter
\def\@seccntformat#1{APPENDIX \csname the#1\endcsname.~}
\makeatother

%------------------------------------------------------------------------------
% If you need to make one (or more) appendix (appendices), place them here as
% sections
%------------------------------------------------------------------------------
\section{HOW TO MAKE APPENDICES}
\label{app::a}

This is a placeholder for my first appendix

\section{OTHER APPENDIX STUFF}
\label{app::b}

This is a placeholder for my second appendix

\end{document}

