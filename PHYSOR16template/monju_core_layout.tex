%------------------------------------------------------------------------------
% PGF/TikZ diagram of the Monju core. The basic approach is as follows:
% - the SAs are represented by nodes with a hexagonal shape
% - the pitch between SAs determines the size of the hexagons (using the inner
%   sep parameter)
% - for the major SA types there is a predefined color
% - the core has 60 degree symmetry, so use the rotation capability of PGF/TikZ
%   to determine the location of each SA
% - draw the diagram of the basic regions first
% - then overlay the "special cases"
% - LaTeX is not very good for calculations, and as a result, this kind of dia-
%   gram takes some time to compile (for example, each SA position must be
%   calculated, then a hexagon needs to be calculated around that position, etc)
%------------------------------------------------------------------------------

\begin{tikzpicture}[scale=1,
  default hexagon/.style={regular polygon, regular polygon sides=6, draw, 
  inner sep=0.07cm, fill=white, draw=white},
  reflector/.style={default hexagon, fill=blue!50!white, draw=blue!50!black},
  blanket/.style={default hexagon, fill=green!50!white, draw=green!50!black},
  outer fuel/.style={default hexagon, fill=red!50!white, draw=red!50!black},
  inner fuel/.style={default hexagon, fill=orange!70!white, draw=orange!50!black},
  ccr/.style={default hexagon, fill=yellow!90!white, draw=yellow!50!black},
  bcr/.style={default hexagon, fill=pink!90!white, draw=pink!50!black},
  fcr/.style={default hexagon, fill=purple!90!white, draw=purple!50!black},
  source/.style={default hexagon, fill=gray!50!white, draw=gray!50!black}
  ]

  \pgfmathsetmacro{\pitch}{0.2}

  %----------------------------------------------------------------------------
  % First ring
  %----------------------------------------------------------------------------
  \foreach \i in {0,...,5}{
    \node[inner fuel] at (\i * 60 + 30: \pitch) {} ;
  }
  
  %----------------------------------------------------------------------------
  % Remainder of inner fuel region
  %----------------------------------------------------------------------------
  \foreach \r in {2,...,6}{
    \foreach \i in {0,...,5}{
      \node[inner fuel] at (\i * 60 + 30: \r * \pitch) {} ;
      \pgfmathsetmacro{\rmo}{\r-1}
      \foreach \rr in {1,...,\rmo}{
        \draw ++(\i * 60 + 30: \r * \pitch)  +(\i * 60 + 150: \rr * \pitch) 
          node[inner fuel] {} ;      
      }
    }
  }

  %----------------------------------------------------------------------------
  % Outer fuel region
  %----------------------------------------------------------------------------
  \foreach \r in {7,8}{
    \foreach \i in {0,...,5}{
      \node[outer fuel] at (\i * 60 + 30: \r * \pitch) {} ;
      \pgfmathsetmacro{\rmo}{\r-1}
      \foreach \rr in {1,...,\rmo}{
        \draw ++(\i * 60 + 30: \r * \pitch)  +(\i * 60 + 150: \rr * \pitch) 
          node[outer fuel] {} ;      
      }
    }
  }

  %----------------------------------------------------------------------------
  % Blanket region
  %----------------------------------------------------------------------------
  \foreach \r in {9,...,11}{
    \foreach \i in {0,...,5}{
      \node[blanket] at (\i * 60 + 30: \r * \pitch) {} ;
      \pgfmathsetmacro{\rmo}{\r-1}
      \foreach \rr in {1,...,\rmo}{
        \draw ++(\i * 60 + 30: \r * \pitch)  +(\i * 60 + 150: \rr * \pitch) 
          node[blanket] {} ;      
      }
    }
  }

  %----------------------------------------------------------------------------
  % Reflector region
  %----------------------------------------------------------------------------
  \foreach \r in {12,...,14}{
    \foreach \i in {0,...,5}{
      \node[reflector] at (\i * 60 + 30: \r * \pitch) {} ;
      \pgfmathsetmacro{\rmo}{\r-1}
      \foreach \rr in {1,...,\rmo}{
        \draw ++(\i * 60 + 30: \r * \pitch)  +(\i * 60 + 150: \rr * \pitch) 
          node[reflector] {} ;      
      }
    }
  }

  %----------------------------------------------------------------------------
  % Reflector region last ring, missing 6 assemblies
  %----------------------------------------------------------------------------
  \foreach \r in {15,...,15}{
    \foreach \i in {0,...,5}{
      \pgfmathsetmacro{\rmo}{\r-1}
      \foreach \rr in {1,...,\rmo}{
        \draw ++(\i * 60 + 30: \r * \pitch)  +(\i * 60 + 150: \rr * \pitch) 
          node[reflector] {} ;      
      }
    }
  }

  %----------------------------------------------------------------------------
  % Neutron sources (2 SAs)
  %----------------------------------------------------------------------------
  \node[source] at (90:  9 *\pitch) {} ;
  \node[source] at (270:  9 *\pitch) {} ;

  %----------------------------------------------------------------------------
  % 6 reflectors in blanket region
  %----------------------------------------------------------------------------
  \foreach \i in {0,...,5}{
    \node[reflector] at (\i * 60 + 30: 11 * \pitch) {} ;
  }

  %----------------------------------------------------------------------------
  % 10 coarse control rods
  %----------------------------------------------------------------------------
  \node[ccr] at (0,0) {} ;
  \draw ++(90: 2 * \pitch)  +(30: \pitch) node[ccr] {} ; 
  \draw ++(90: 4 * \pitch)  +(30: 2 * \pitch) node[ccr] {} ; 
  \draw ++(90: -1 * \pitch)  +(-30: 2 * \pitch) node[ccr] {} ; 
  \draw ++(90: -2 * \pitch)  +(-30: 4 * \pitch) node[ccr] {} ; 
  \draw ++(90: -2 * \pitch)  +(150: 3 * \pitch) node[ccr] {} ; 
  \draw ++(90: -4 * \pitch)  +(150: 6 * \pitch) node[ccr] {} ; 
  \draw ++(90: 2 * \pitch)  +(150: 4 * \pitch) node[ccr] {} ; 
  \draw ++(90: -2 * \pitch)  +(30: 6 * \pitch) node[ccr] {} ; 
  \draw ++(90: -6 * \pitch)  +(150: 2 * \pitch) node[ccr] {} ; 

  %----------------------------------------------------------------------------
  % 6 backup rods
  %----------------------------------------------------------------------------
  \draw ++(90: 4 * \pitch)  +(150: \pitch) node[bcr] {} ; 
  \draw ++(90: \pitch)  +(150: 2 * \pitch) node[bcr] {} ; 
  \draw ++(90: -3 * \pitch)  +(150: \pitch) node[bcr] {} ; 
  \draw ++(90: -5 * \pitch)  +(150: 4 * \pitch) node[bcr] {} ; 
  \draw ++(90: -1 * \pitch)  +(30: 3 * \pitch) node[bcr] {} ; 
  \draw ++(90: -4 * \pitch)  +(30: 5 * \pitch) node[bcr] {} ; 

  %----------------------------------------------------------------------------
  % 3 fine control rods
  %----------------------------------------------------------------------------
  \draw ++(90: 1 * \pitch)  +(30: 4 * \pitch) node[fcr] {} ; 
  \draw ++(90: -1 * \pitch)  +(150: 5 * \pitch) node[fcr] {} ; 
  \draw ++(90: -5 * \pitch)  +(30: \pitch) node[fcr] {} ; 

  \node[](AA) at ( -3.3,  0 ) {\SI{180}{\degree}};
  \node[](BB) at (  3.3,  0 ) {\SI{0}{\degree}} ;
  \node[](CC) at (   0, 3.6 ) {\SI{90}{\degree}}  ;
  \node[](DD) at (   0,-3.6 ) {\SI{270}{\degree}} ;
  \draw[thick] (AA) -- (BB) ;
  \draw[thick] (CC) -- (DD) ;
\end{tikzpicture}

